%-*-coding:utf-8;-*-
\documentclass[14pt]{beamer}
\usepackage[T1]{fontenc}
\usepackage[utf8]{inputenc}
%\usepackage[usenames,dvipsnames,svgnames,table]{xcolor}
\definecolor{links}{HTML}{2A1B81}
\hypersetup{colorlinks,linkcolor=,urlcolor=links}
\usetheme[numbering=none]{metropolis}           % Use metropolis theme
\setsansfont{Gentium}
\hyphenation{me-mo-ra-bi-li-um}
\title{De corpore dramatum Latinorum in Croatia actorum restituendo}
\date{die 31 Julii 2018}
\author{Neven Jovanović}
\institute{Facultas philosophica Universitatis Zagrabiensis}
\begin{document}
  \maketitle
  


\begin{frame}
\tableofcontents
\end{frame}

\section{Prooemium: De dramatum in Croatia ignorantia et notitia}

\begin{frame}
  In Croatia ab anno 1525 usque ad annum 1805 plus quam sescentae actiones Latinas in scaena productae sunt.
\end{frame}

\begin{frame}
  Ubi id factum sit, et quorum ope?
  Qui libri impressi et manu scripti exsistunt?
  Qui dramatum tituli?
  Quid spectatores de actionibus iudicaverint?
\end{frame}

\begin{frame}
  Quare de dramatibus Latinis in Croatia tam diu nihil sciverimus, seu nihil scire voluerimus?
\end{frame}

\begin{frame}
Narratio: de Croatia, quae terra et ubi?
\end{frame}

\begin{frame}
Narratio: dramatum numerus et frequentia, productionum tempora et loca
\end{frame}

\begin{frame}
Narratio: ubi actiones ipsa sua absentia praefulgeant?
\end{frame}

\begin{frame}
Argumentatio: unde et quibus modis de dramatibus notitias hauriamus?
\end{frame}

\begin{frame}
Argumentatio: quomodo distinxerimus res de quibus fabulae agantur?
\end{frame}

\begin{frame}
  Argumentatio: exempla notitiarum typica et curiosa; exempla fabularum repetitarum
  
\end{frame}

\begin{frame}
  Peroratio: conclusiones quasdam, et quid porro faciendum et investigandum sit?
  
\end{frame}

\section{Prooemium: De ignorantiae quattuor causis}

\begin{frame}
  textus dramatum deperditi
  dramata in scholis exhibita
  deest Auctor
  religio et ecclesia suspectae
  
\end{frame}

\section{Narratio: De Croatia saeculis XVI-XIX}

{
    \usebackgroundtemplate{\includegraphics[width=\paperwidth]{img/centraleu2.jpg}}
    \setbeamertemplate{navigation symbols}{}
    \begin{frame}[plain]
    \end{frame}
    }

{
    \usebackgroundtemplate{\includegraphics[width=\paperwidth]{img/illyricumhod.png}}
    \setbeamertemplate{navigation symbols}{}
    \begin{frame}[plain]
    \end{frame}
    }


\begin{frame}[Croatia inter tres aut quattuor dominationes divisa]

  Res publica Venetorum
  Provinciae hereditariae regni Austriaci
  Ottomanicum imperium
  Res publica Ragusina
  Regnum Croatiae, Regnum Slavoniae

\end{frame}


\begin{frame}{Societas Jesu in Croatia}

\alert{1559} venerunt Ragusam

\alert{1604–1698} fundaverunt collegia Ragusae, Zagrabiae, Fluminis S. Viti (Rijeka), Varasdini, Posegae

\alert{1773} suppressio ordinis
\end{frame}

{
    \usebackgroundtemplate{\includegraphics[width=\paperwidth]{img/eichler.jpeg}}
    \setbeamertemplate{navigation symbols}{}
    \begin{frame}[plain]
    \end{frame}
    }



\section{Narratio: quid, quando, ubi, quot?}




\section{Argumentatio: unde et quomodo notitias hauserimus?}

\begin{frame}[Bibliographiae]

Martina Petranović and Lucija Ljubić. \emph{Repertoar hrvatskih kazališta : Knjiga peta : Deskriptivna obrada važnijih predstava na hrvatskom jeziku i izvedbi na stranim jezicima hrvatskih izvođača do 1840. godine}, Zagreb 2012.

Staud, Géza. \emph{A magyarországi jezsuita iskolai színjátékok forrásai, III. : 1561-1773, Fontes ludorum scenicorum in scholis S. J. Hungariae, pars tertia}, Budapest 1988.

\end{frame}
\begin{frame}[Historiae]

Fancev, Franjo, Građa za povijest školskog i književnog rada
isusovačkoga kolegija u Zagrebu. Starine. Knj. 37. Jugoslavenska
akademija znanosti i umjetnosti, Zagreb 1934.

Fancev, Franjo, Građa za povijest školskog i književnog rada
isusovačkoga kolegija u Zagrebu. Starine. Knj. 38. Jugoslavenska
akademija znanosti i umjetnosti, Zagreb 1937.

\end{frame}

\section{Methods}

\begin{frame}

An XML file holding records of performances.

An XML database with recorded searches.

A version-controlled repository with data, scripts, and documentation: \href{https://github.com/nevenjovanovic/croaladrama}{github.com/nevenjovanovic/croaladrama}.

The repository is freely available, can be updated, enhanced, integrated into larger collections.

\end{frame}

\section{Findings}

\begin{frame}{Findings}

The database holds records on \alert{686} performances in 1607–1805.

There are \alert{19} performances (2.6 percent) whose titles have to do with Japan.

\end{frame}

\begin{frame}{Findings}

The plays on Japan were performed during the 133 years between 1628 and 1761. 

They were performed in Jesuit colleges of the four cities:\\
\alert{Zagreb} (established 1606)\\
\alert{Varaždin} (Jesuits present from 1632, college established 1678)\\
\alert{Rijeka} (college established 1627)\\
\alert{Požega} (college established 1698).

\end{frame}

{
    \usebackgroundtemplate{\includegraphics[width=\paperwidth]{img/illyricum_urb.png}}
    \setbeamertemplate{navigation symbols}{}
    \begin{frame}[plain]
    \end{frame}
    }





  \maketitle


\end{document}

